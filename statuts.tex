\documentclass[12pt]{article}
\usepackage[top=2cm, bottom=3cm, left=3cm, right=3cm]{geometry}
\usepackage[utf8]{inputenc}
\usepackage{charter}
\usepackage[T1]{fontenc}
\usepackage[french]{babel}
\usepackage{hyperref}
\usepackage{parskip}
\usepackage{comment}

\begin{comment}
Informations sur le formatage du document

Section et sous-sections:
Chaque section est définie par un \section{} et chaque sous-section par un \subsection{}.
Toute déclaration d'une section ou d'une sous-section doit être suivie d'un \label{} pour pouvoir être référencée.
Le format d'un label est le suivant: sec:<nom de la section> pour une section et sec:<nom de la section>:<nom de la sous-section> pour une sous-section.
Les sections sont référencées à l'aide de \autoref{<nom de la référence>}.
\end{comment}


% Informations sur l'association
\newcommand{\asso}{Nom de l'asso}
\newcommand{\dateCreation}{1 juin 2022}

\title{Statuts \asso}
\author{}
\date{\today}

\begin{document}
\maketitle

\vspace*{10cm}

Dernière modification: \today.

\section{Préambule}
\label{sec:preambule}

Il est fondé, depuis le \dateCreation, entre les adhérents aux présents statuts, une association régie par la loi du 1er juillet 1901 et le décret du 16 août 1901 ayant pour titre :\\
\\
\textbf{\asso}\\
\\
Ci-après dénommée \asso, et régie par les présents statuts.\\
Sa durée est illimitée.

\section{Objet et moyens}
\label{sec:objet}

\asso~ est une association à but non lucratif de solidarité locale et internationale.
Il s'agit d'aider des populations par la récolte de fonds, par le bénévolat spontané et local ou encore par le
bénévolat à l’étranger dans le cadre d’un projet mené par une association autour de l'éducation, la santé,
l'environnement ou la construction.

\section{Siège social}
\label{sec:siege}

Le siège social de \asso~ est fixé dans les locaux de TELECOM Nancy, à ce jour :\\
193, avenue Paul Muller\\
BP 90172\\
54600 VILLERS-LES-NANCY CEDEX\\
Il peut être modifié sur simple décision du Conseil d’Administration.

\section{Exercice social}
\label{sec:exercice}

L’exercice social de \asso~commence au premier jour ouvré du mois de Janvier et se termine la veille du
premier jour ouvré du mois de Janvier de l’année suivante.

\section{Composition}
\label{sec:composition}

\asso~se compose de 6 catégories de membres :

\begin{itemize}
    \item \textbf{Membres du Bureau}, membres du Conseil d’Administration en charge de la gestion stratégique de l’association.
    \item \textbf{Membres du Conseil d’Administration}, administrateurs, membres adhérents en charge de la gestion de l’association.
    \item \textbf{Membres d’honneur}, personnes physiques ou morales ayant rendu des services signalés à l’Association, dispensés de cotisations.
    \item \textbf{Membres ponctuels}, toute personne souhaitant participer de manière ponctuelle aux activités de l’association, sous réserve de non refus par le Conseil d’Administration. Les membres ponctuels ne sont membres que pendant la durée de ladite manifestation et n’ont pas le droit de participer et de voter aux assemblées générales.
    \item \textbf{Membres bienfaiteurs}, toute personne ayant fait un don à l’association, sous réserve d’acceptation par le Conseil d’Administration. Ils ont droit de participation mais pas de vote aux Assemblées Générales.
    \item \textbf{Membres adhérents}, toute personne souhaitant participer aux activités de l’association qui pourront avoir accès à l’ensemble des services et activités proposées par l’association sous réserve de non refus par le Conseil d’Administration. Les membres adhérents ont droit de participation et de vote lors de l’Assemblée Générale.
\end{itemize}



\section{Admission}
\label{sec:admission}

\subsection{Membres adhérents}
\label{sec:admission:adhérents}

L’admission d’un membre adhérent se fait, à tout moment, sur simple inscription de sa part auprès du
Secrétaire Général. Les conditions d’adhésion sont détaillées dans le Règlement Intérieur.

\subsection{Administrateurs}
\label{sec:admission:administrateurs}

Les nouveaux administrateurs sont désignés, lors d’une Assemblée Générale Ordinaire ou Extraordinaire,
parmi des candidats élèves de TELECOM Nancy en 1ère ou 2ème année par le Conseil d’Administration au vote
à la majorité simple.
Les admissions seront effectives après signature du bulletin d’Adhésion et après acquittement de la
cotisation.

\section{Ressources}
\label{sec:ressources}

\subsection{Cotisations}
\label{sec:ressources:cotisations}

La cotisation est définie dans l’article 1.2 du Règlement Intérieur.

\subsection{Autres ressources financières}
\label{sec:ressources:autres}
\asso pourra percevoir des ressources financières dans les conditions fixées par la loi, par les moyens suivants :
\begin{itemize}
    \item Des subventions de l'Etat, des régions, des départements et des communes et par des organismes tiers.
    \item Des bénéfices réalisés lors de diverses manifestations/événements.
    \item Des partenariats privés.
    \item Des dons privés.
\end{itemize}



\section{Radiation-Démission}
\label{sec:radiation}

\subsection{Administrateur}
\label{sec:radiation:admin}

La qualité d’administrateur se perd par la démission, le décès ou au plus tard un an après sa prise de fonction,
sauf en cas de décision contraire du Conseil d’Administration.

\subsection{Membre}
\label{sec:radiation:membre}

La qualité de membre se perd automatiquement par la démission ou le décès et, pour les membres du bureau
par le non-renouvellement du certificat de scolarité et la perte de la qualité d’étudiant. Les modalités
concernant la cotisation sont spécifiées dans le Règlement Intérieur article 1.2.

\subsection{Tout type de membre}
\label{sec:radiation:toutmembre}
Tout membre de l’association peut être radié selon les modalités suivantes :
Le Conseil d’Administration peut voter à la majorité renforcée des deux tiers (ou la majorité moins la personne
concernée si celle-ci fait partie du Conseil d’Administration) le lancement d’une procédure de radiation envers
un des membres de l’association.

\subsection{La procédure de radiation}
\label{sec:radiation:procedure}

Le Conseil d’Administration devra se réunir au plus tôt une semaine après que le membre concerné ait été
notifié de la procédure. Le Conseil d’Administration devra auditionner le membre concerné. Le vote se fera à
la majorité renforcée des deux tiers sans la présence du membre concerné.


\section{Bureau}
\label{sec:bureau}

\subsection{Composition}
\label{sec:bureau:composition}

Le Conseil d’Administration élit en son sein pour un an maximum et à la majorité absolue, un Bureau composé
au minimum des personnes distinctes suivantes :

\begin{itemize}
    \item un Président
    \item un Trésorier
    \item un Secrétaire Général
    \item un Vice-Président
\end{itemize}

Le Bureau peut définir des postes supplémentaires au Conseil d’Administration à l’unanimité des membres
du Bureau, le vote s’effectuant à bulletin secret ou à main levée après consultation des membres du Bureau.
L’élection pour ce poste supplémentaire s’effectuera en fonction du point 9.6 des Statuts.
Il peut aussi en supprimer selon les mêmes modalités dans le cas où le poste est vacant. Dans le cas où il
souhaite à l’unanimité en supprimer un et que le poste est pourvu, le vote de suppression de ce poste devra
se faire en réunion du Conseil d’Administration avec un vote à la majorité des deux tiers. Le vote s’effectue à
bulletin secret ou à main levée après consultation des membres du Conseil d’Administration.


\subsection{Attributions}
\label{sec:bureau:attributions}

Le Bureau établit la stratégie de l’année et applique la politique définie par le Conseil d’Administration.

\subsection{Responsabilité}
\label{sec:bureau:responsabilité}

Le Président est responsable de la gestion morale de l’association. Il est le seul habilité à signer les documents
engageant l’Association. Il ordonne les dépenses avec l’accord nécessaire du Trésorier.\\
Le Président est habilité à effectuer les ordres de paiement.\\
Le Président peut donner délégation de ses pouvoirs à tout membre du Conseil d’Administration à l’exception
du Trésorier, pour au maximum une semaine. Pour une délégation de plus d’une semaine, le choix du
Président devra être approuvé à la majorité simple par le Conseil d’Administration.\\
Le Trésorier est responsable de la gestion comptable et financière de l’association. Il assure le recouvrement
des cotisations et des ressources de toute nature.\\
Dans le cas où au moins un administrateur distinct du Trésorier est affecté à la saisie comptable, le Trésorier
est également habilité à effectuer les ordres de paiement.\\


Le Trésorier peut donner délégation de ses pouvoirs à tout administrateur à l’exception du Président et de
toute personne affectée à la saisie comptable, pour au maximum une semaine. Pour une délégation de plus
d’une semaine, le choix du Trésorier devra être approuvé à la majorité simple par le Conseil d’Administration.

\subsection{Réunions}
\label{sec:bureau:reunions}

Le Bureau se réunit sur convocation du Président, ou à la demande de deux de ses membres. Les décisions
sont prises à la majorité simple, la voix du Président étant prépondérante en cas d’égalité.
Le Bureau se réunit au minimum trois fois par an.


\subsection{Elections}
\label{sec:bureau:elections}

Le Bureau est élu chaque année durant l’Assemblée Générale Ordinaire ou Extraordinaire pendant laquelle
le Conseil d’Administration est élu. Cette élection se fait à la majorité simple, soit à bulletin secret, soit à
main levée après consultation de l’assemblée par le président de séance.
Si un candidat est le seul à se présenter pour un poste, il lui suffit d’une seule voix pour être élu à ce poste. Si
plusieurs candidatures sont déposées mais qu’aucun candidat n’obtient la majorité simple, alors un
deuxième tour est organisé. Est élu le candidat qui obtient le plus grand nombre de voix. En cas d’égalité, la
voix du président sortant ou entrant dès lors qu’il a été élu est prépondérante.
L’élection se fait poste par poste, dans l’ordre suivant : Président, Trésorier, Secrétaire, Vice-Président. Dès
lors qu’un candidat est élu à un poste, ses autres candidatures sont supprimées. Dans le cas où un poste du
Bureau est vacant à l’issue de l’Assemblée Générale Ordinaire ou Extraordinaire, se référer au point 9.6 des
Statuts.\\
Les postulants doivent faire connaître leur candidature trois jours au moins avant l’élection, une même
personne pouvant se présenter à différents postes.


\subsection{Vacance}
\label{sec:bureau:vacance}

En cas de vacance d’un poste du Bureau, le Conseil d’Administration cherche un remplaçant.
Si un membre du Conseil d’Administration est élu, le Conseil d’Administration procède au remplacement
lors de sa prochaine réunion. Il y a alors vacance d’un poste au Conseil d’Administration. Il faut donc se
référer à l’article 10 alinéa 5.
Sinon, le remplaçant est élu lors d’une Assemblée Générale Extraordinaire entre tous les membres. Cette
personne entre dans une période d'essai de deux semaines à l'issue de laquelle le Conseil d’Administration
vote pour sa prise de fonction effective ou non.
Si aucun remplaçant n’est trouvé, un vote aura lieu au sein du Conseil d’Administration pour désigner une
personne du Conseil d’Administration pour occuper le poste vacant. L’occupation de ce poste est limitée à 2
mois à partir du vote.
Le Conseil d’Administration continue de chercher un remplaçant.
Le vote du remplacement se fait à la majorité simple, à bulletin secret ou à main levée après consultation de
l'assemblée par le président de séance.
Les postes du Bureau doivent être occupés par 4 personnes différentes dans un cadre de remplacement.


\subsection{Mobilité}
\label{sec:bureau:mobilite}

Dans le cas d’une mobilité de plus de trois mois, un vote sera effectué au sein du Conseil d’Administration afin
de déterminer si cette personne doit démissionner de son poste. Le vote s’effectuera à la majorité simple à
bulletin secret ou à main levée après consultation des membres du Conseil d’Administration. Le membre du
Bureau se doit de présenter sa démission au Conseil d’Administration dans le mois précédent son départ, au
minimum une semaine avant, avec une prise d’effet le jour de son départ.
Durant cette période, le Conseil d’Administration procède au remplacement du membre concerné en se
référant à l’article 9.6 des présents statuts.


\section{Conseil d’Administration}
\label{sec:conseil}

\subsection{Composition et structure}
\label{sec:conseil:composition}

Le Conseil d’Administration est composé des administrateurs et est présidé par le Président ou le Vice-
président. En cas d’absence de ces membres, un administrateur sera désigné par convention tacite en début
de séance.
Le Conseil d’Administration est constitué des membres du Bureau et des responsables des différents pôles de
l’association, définis dans le Règlement Intérieur paragraphe 2 alinéa 1 article 2.
Le Conseil d’Administration doit contenir au minimum 8 membres dont les postes sont spécifiés dans le
Règlement Intérieur article 2.1.2 et peut contenir au maximum 11 membres.
Lors de sa première réunion, le Conseil d’Administration définit son organisation : son organigramme,
notamment en votant les membres du Bureau, et son mode de fonctionnement.

\subsection{Attributions}
\label{sec:conseil:attributions}

Le Conseil d’Administration est investi des pouvoirs de gestion les plus étendus, peut agir en toutes
circonstances au nom de l’association et prendre toute décision relative à tout acte d’administration, de
disposition ou de gestion.

\subsection{Election}
\label{sec:conseil:election}

Le Conseil d’Administration est élu poste par poste chaque année, lors de l’Assemblée Générale Ordinaire ou
Extraordinaire, au cours d’un vote à bulletin secret ou à main levée après consultation de l’assemblée par le
Président, à la majorité simple des membres adhérents et des membres d’honneurs. Tout membre adhérent
de l’association a le droit de postuler aux postes vacants.
Si un candidat est le seul à se présenter pour un poste, il lui suffit d’une seule voix pour être élu à ce poste. Si
plusieurs candidatures sont déposées mais qu’aucun candidat n’obtient la majorité simple, alors un
deuxième tour est organisé. Est élu le candidat qui obtient le plus grand nombre de voix. En cas d’égalité, la
voix du président entrant est prépondérante.
L’élection du Conseil d’Administration se fait à la suite de l’élection du Bureau. Elle se fait poste par poste en
désignant pour chaque poste le candidat choisit, dans l’ordre défini par le Président entrant. Si aucun ordre
n’est choisi, se référer à l’ordre de l’article 2.1.2 du Règlement Intérieur. Dès lors qu’un candidat est élu à un
poste, ses autres candidatures sont supprimées.
Dans le cas où un poste obligatoire du Conseil d’Administration (voir l’article 2.1.2 du Règlement Intérieur)
est vacant à l’issue de l’Assemblée Générale Ordinaire ou Extraordinaire, se référer au point 10.5 des
Statuts.

\subsection{Réunions}
\label{sec:conseil:reunions}

Le Conseil d’Administration se réunit sur convocation du Président ou à la demande du tiers de ses membres.
Il est possible d’assister au Conseil d’Administration par visioconférence.
Hors période de congés, le Conseil d’Administration se tient au moins une fois par mois.
Le quorum est fixé à deux tiers de ses membres. Dans le cas où le quorum n’est pas atteint, ce dernier est fixé
à la moitié puis au quart pour les deux Conseils d’Administration suivants.
Les décisions se prennent à la majorité simple des membres présents ou ayant donné procuration à un
membre présent, sauf cas particuliers mentionnés dans le Règlement Intérieur. La procuration doit faire
l’objet d’un document écrit et signé ou de l’envoi d’un courrier électronique au Président de séance.
Lors des votes, la voix du président de séance est prépondérante en cas d’égalité. L’ordre du jour est réglé par
le président de séance, en concertation avec le Secrétaire Général [Seulement si le président de séance n’est
pas le Secrétaire Général].
Chaque membre du Conseil d’Administration a la possibilité de faire apparaître des points à l’ordre du jour.
Lors de sa réunion, le Conseil d’Administration peut exiger du Trésorier la communication de l’état de la
trésorerie.
Un compte-rendu de réunion est rédigé par le Secrétaire Général et archivé dans le cahier d’Association. Il est
signé et paraphé par le Président et le Secrétaire Général.

\subsection{Vacance}
\label{sec:conseil:vacance}

En cas de vacance d’un poste obligatoire du Conseil d’Administration, et si ce poste n’est pas un poste du
Bureau (dans ce cas, se référer à la \autoref{sec:bureau:vacance}), le Conseil d’Administration procède au remplacement
du poste vacant.
Le Conseil d’Administration élit à la majorité simple un membre adhérent auquel est proposé le poste
vacant. Le vote se fait à bulletin secret ou à main levée après consultation de l’assemblée par le président de
séance. Dans le cas d’une acceptation de ce dernier, il prend la position vacante au Conseil
d’Administration.
Si aucun remplaçant n’est trouvé, un vote aura lieu au sein du Conseil d’Administration pour désigner une
personne du Conseil d’Administration pour occuper le poste vacant. L’occupation de ce poste est limitée à 2
mois à partir du vote.
Le Conseil d’Administration continue de chercher un remplaçant.


\section{Assemblée Générale Ordinaire}
\label{sec:ag}

\subsection{Composition}
\label{sec:ag:composition}

La composition de l’Assemblée Générale est définie dans le Règlement Intérieur à l’article 2.3.2.

\subsection{Attributions}
\label{sec:ag:attributions}

L’Assemblée Générale Ordinaire a notamment pour but le renouvellement du Conseil d’Administration.
Le Président expose la situation morale et financière d’\asso. Il rend compte des actions qu’il a menées
pendant son mandat et fournit un bilan moral de ce dernier (pouvant prendre la forme d’un rapport d’activité
écrit).
Le Trésorier rend compte de la gestion financière d’\asso~et présente les comptes (bilan, compte de
résultat et annexes).

Il est alors procédé au vote des quitus du Président et du Trésorier.
A l’issue du vote des quitus, l’élection du nouveau Conseil d’Administration se déroule conformément à
l’article 10 alinéa 3.

\subsection{Réunions}
\label{sec:ag:reunions}

L’Assemblée Générale Ordinaire se réunit annuellement sur convocation du Président.
Son ordre du jour est réglé par le Conseil d’Administration dans les deux semaines qui la précède.
Seuls les membres de l’assemblée présents ou ayant donné procuration à un autre membre de l’assemblée
présent ont le droit de vote. La procuration doit faire l’objet d’un document écrit et signé ou de l’envoi d’un
courrier électronique au Président.
Le quorum est fixé à deux tiers des votants. Dans le cas où le quorum n’est pas atteint, l’Assemblée Générale
Ordinaire est nulle et sans effet. Le Président doit alors convoquer une nouvelle Assemblée Générale
Ordinaire dans un délai de deux semaines. Cette nouvelle Assemblée Générale Ordinaire pourra délibérer de
façon valable quel que soit le nombre de votants.
Les votes se font à bulletin secret ou à main levée après consultation de l’assemblée par le président de
séance. Les décisions se prennent à la majorité simple. En cas d’égalité, la voix du Président est prépondérante
et sera annoncée à haute voix.
Un procès-verbal d’Assemblée Générale est rédigé par le Secrétaire Général, ou un secrétaire de séance
nommé par le Président en début de séance, et archivé dans le cahier d’Association. Il est signé et paraphé
par le président et le secrétaire de séance.

\subsection{Procurations}
\label{sec:ag:procurations}

Seuls les membres présents ou ayant donné procuration à un membre présent ont le droit de vote. La
procuration doit faire l’objet d’un document écrit et signé ou de l’envoi d’un courrier électronique au
Président. Un membre présent ne peut pas cumuler plus de deux procurations.

\section{Assemblée Générale Extraordinaire}
\label{sec:ago}

\subsection{Composition}
\label{sec:ago:composition}
L’Assemblée Générale Extraordinaire est de composition identique à celle de l’Assemblée Générale Ordinaire.

\subsection{Attributions}
\label{sec:ago:attributions}
L’Assemblée Générale Extraordinaire permet de voter la modification des Statuts de l’Association, ainsi que
la dissolution de l’Association.

\subsection{Réunions}
\label{sec:ago:reunions}
L’Assemblée Générale Extraordinaire se réunit sur convocation du Président ou de deux des membres du
Bureau ou de 50\% du Conseil d’Administration et 2/3 des membres de l’association.
Son ordre du jour est réglé par le Conseil d’Administration dans les 2 semaines qui la précèdent.
Seuls les membres présents ou ayant donné procuration à un membre présent ont le droit de vote. La
procuration doit faire l’objet d’un document écrit et signé ou de l’envoi d’un courrier électronique au
Président.
Le quorum est fixé à deux tiers des votants. Dans le cas où le quorum n’est pas atteint, l’Assemblée Générale
Extraordinaire est nulle et sans effet.

Le Président doit alors convoquer une nouvelle Assemblée Générale Extraordinaire dans un délai de deux
semaines. Cette nouvelle Assemblée Générale Extraordinaire pourra délibérer de façon valable quel que soit
le nombre de votants.
Les votes se font à bulletin secret ou à main levée après consultation de l’assemblée par le président de
séance. Les décisions se prennent à la majorité simple, la voix du Président étant prépondérante en cas
d’égalité et sera annoncée à haute voix.

\subsection{Procurations}
\label{sec:ago:procurations}
Seuls les membres présents ou ayant donné procuration à un membre ont le droit de vote. La procuration doit
faire l’objet d’un document écrit et signé ou de l’envoi d’un courrier électronique au Président. Un membre
présent ne peut pas cumuler plus de deux procurations.

\section{Règlement Intérieur}
\label{sec:ago:reglement}

Un Règlement Intérieur est établi par le Conseil d’Administration.
Ce Règlement Intérieur est destiné à compléter les présents Statuts, notamment en ce qui concerne
l’administration interne de l’association.
Celui-ci peut être modifié par vote à la majorité simple du Conseil d’Administration.

\section{Modification des Statuts}
\label{sec:ago:statuts}
Elle est proposée par le Conseil d’Administration lors d’une Assemblée Générale Ordinaire, ou lors d’une
Assemblée Générale Extraordinaire convoquée pour la circonstance.
Les membres de l’association représentant au moins 20\% des membres adhérents peuvent demander
l’inscription à l’ordre du jour d’une proposition de Statuts.
Cette proposition, sous réserve qu’elle soit notifiée par le Conseil d’Administration au moins sept jours avant
la tenue de l’Assemblée Générale, fera l’objet d’un vote de l’Assemblée Générale.

\section{Dissolution de l’association}
\label{sec:ago:dissolution}

La dissolution est décidée par le Conseil d’Administration, puis soumise à l’approbation de l’Assemblée
Générale Extraordinaire.
L’Assemblée Générale Extraordinaire doit alors désigner un liquidateur.

\end{document}